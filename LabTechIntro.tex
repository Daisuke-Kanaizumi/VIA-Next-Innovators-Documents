% !TEX TS-program = platex
%!TEX encoding = Shift-JIS
\documentclass[dvipdfmx,10pt]{beamer}


\ifnum 42146=\euc"A4A2 %"
\AtBeginShipoutFirst{\special{pdf:tounicode EUC-UCS2}}%
\else
\AtBeginShipoutFirst{\special{pdf:tounicode 90ms-RKSJ-UCS2}}%
\fi

\usefonttheme{professionalfonts}
\setbeamertemplate{navigation symbols}{}
\setbeamertemplate{footline}[frame number]
\setbeamertemplate{bibliography item}[text]
\usepackage{listings}
\usepackage{amsmath,amsfonts,amsthm,latexsym}
\usepackage{lmodern}
\usepackage{pxjahyper}
\usepackage{cases,color,comment}
\newcommand{\bb}{\begin{block}}
\newcommand{\eb}{\end{block}}
\newcommand{\ft}{\frametitle}
\newcommand{\ssec}{\subsection}
\newcommand{\sss}{\subsubsection}
\newcommand{\insec}{\insertsection}
\newcommand{\iss}{\insertsubsection}
\newcommand{\isss}{\insertsubsubsection}
\newcommand{\bit}{\begin{itemize}}
\newcommand{\eit}{\end{itemize}}

\usetheme{Berkeley}
\usecolortheme{albatross}
\mathversion{bold}
\title{Introduction to LabTech}
\subtitle{How to search \& save things on the Internet\\
-Advanced tips for graduate students-
}
\author[VIA Next Innovators Tokyo]{VIA Next Innovators Tokyo\footnote{\texttt{next\_innovators\_tokyo@viaprograms.org}}\\(Authorized Affiliate of VIA Programs\footnote{\texttt{http://viaprograms.org/}})}
\date[November 2019, @ Learning Cafe]{November 2019, at our Learning Cafe}
\begin{document}
\begin{frame}
\titlepage
\end{frame}
\begin{frame}{Outline}
%https://tex.stackexchange.com/questions/231128/beamer-highlighting-subsubsections-in-toc
\setcounter{tocdepth}{3}
\tableofcontents[
sectionstyle=show,
subsectionstyle=show/show,
subsubsectionstyle=show/show/show
]
\end{frame}
\section{Introduction}
\begin{frame}\ft{\insertsection}
\bb{What is LabTech ?}
Laboratory Tech$\to$Made to help your research.
\bit
\item Search \& Save
\item Data Backup
\bit
\item \LaTeX
\item Source Code (C, C++, Python, Julia)
\eit
\item Smarter Browsing on the Internet
\eit
\eb
\end{frame}
\section{Open Access}
\begin{frame}\ft{\insertsection}
\bit
\item Open Access Week is held every October
\item Here are some things related to open access
\eit
\bb{}
\bit
\item arXiv
\item doaj.org
\item PubPeer
\item SpringerOpen
\item Google Chrome/Firefox extensions
\bit
\item Open Access Button
\item Unpaywall
\eit
\item Researcher SNS 
\bit
\item academia.edu
\item ResearchGate
\eit
\eit
\eb
\end{frame}
\section{Search \& Save}
\begin{frame}\ft{\insertsection}
We have many search engines.
\bit
\item Google
\item Bing
\item DuckDuckGo\footnote{\texttt{https://duckduckgo.com/}}
\item Baidu (Used in China)
\item Yandex (Used in Russia)\footnote{\texttt{https://yandex.com/}}
\eit
But there are better options for academic data.
\end{frame}
\subsection{Searching Documents}
\subsubsection{Databases}
\begin{frame}\ft{\insertsubsubsection}
\bit
\item Academia.edu, ResearchGate (researcher version LinkedIn)
\item Academic Search\footnote{\texttt{https://academicsearch.org/}}
\item arXiv (preprint database)
\item Google Scholar
\item SpringerLink (database for Springer publications)
\item ScienceDirect (database for Elsevier publications)
\item Core\footnote{\texttt{https://core.ac.uk/}}, JSTOR, Scopus
\item NSF
\item Semantic Scholar
\item Wiley Online Library
\eit
\end{frame}
\begin{frame}\ft{\insertsubsubsection}
Sometimes SNS are useful.
\bit
\item Unofficial arXiv bots for each areas
\item Springer, Elsevier accounts
\eit
Tells breaking news in each discipline.
\end{frame}
\subsubsection{Well-Known Academic Publishers}
\begin{frame}\ft{\insertsubsubsection}
\bit
\item Springer group
\item Elsevier
\item Academic Press
\item Addison-Wesley
\item CRC Press
\item Dover Publications
\item Walter de Gruyter
\item World Scientific
\item University publishers
\bit
\item Cambridge Univ. Press
\item Oxford Univ. Press
\item Princeton Univ. Press
\eit
\eit
\end{frame}
\subsection{Saving Academic Papers}
\begin{frame}\ft{\insertsubsection}
There are many ways to save things on the Internet.
\bit
\item Dropbox (Dropbox Paper, Dropbox Spaces)
\item Google Drive, Google Keep, Google One
\item Icedrive
\item MEGA\footnote{\texttt{https://mega.nz/}}
\item Pocket (saving bookmarks)\footnote{\texttt{https://getpocket.com/}}
\item Onedrive
\item 4sync
\item 4shared
\eit
But there are better options for academic papers.
\end{frame}
\begin{frame}\ft{\insertsubsection}
These are tools to save academic papers.
\bit
\item Mendeley
\item RefWorks
\item colwiz
\item EndNote
\item ReadCube
\item RefMe
\item Zotero
\eit
\bb{How they work}
\bit
\item Automatically detect publication data
\item Suggest new papers to read
\eit
\eb
\end{frame}
\section{Data Backup}
\begin{frame}\ft{\insertsection}
Unexpected things happen before deadline !!!
\begin{exampleblock}{}
\bit
\item Mistakes found before submission
\item Computer breaks down
\eit
\end{exampleblock}
Don't forget data backup !!!
\end{frame}
\subsection{\LaTeX}
\begin{frame}\ft{\insertsubsection}
There are many ways to save things on the Internet.
\bit
\item Dropbox
\item Google Drive, Google Keep
\item Icedrive
\item MEGA
\item Onedrive
\item 4sync
\item 4shared
\eit
But there are better options for \LaTeX.
\end{frame}
\setbeamerfont{frametitle}{size=\small}
\begin{frame}\ft{\insertsubsection}
\footnotesize
\bb{\small Online \TeX \quad compilers/tools}
\bit
\item Cloud \LaTeX, Overleaf\footnote{\texttt{https://overleaf.com/}}
\item CoCalc\footnote{\texttt{https://cocalc.com/}}, Verbosus/VerbTeX\footnote{\texttt{https://www.verbosus.com/latex.html}}
\item Messenger \LaTeX (Google Chrome extension)
\eit
\eb
\bb{\small \LaTeX$\iff$pictures}
\bit
\item TeXclip\footnote{\texttt{https://texclip.marutank.net/}}, CodeCogs (also generates URL)\footnote{\texttt{https://www.codecogs.com/latex/eqneditor.php}}
\item Mathpix (convert picture to \LaTeX)
\item \LaTeX \quad From Web (Google Chrome extension)
\eit
\eb
Moeditor: \LaTeX \quad version notepad (\LaTeX $+$ HTML)
\normalsize
\end{frame}
\subsection{Source Codes}
\begin{frame}\ft{\insertsubsection}
There are many ways to save things on the Internet.
\bit
\item Dropbox
\item Google Drive, Google Keep
\item Icedrive
\item MEGA
\item Onedrive
\item 4sync
\item 4shared
\eit
But there are better options for source codes.
\end{frame}
\begin{frame}\ft{\insertsubsection}
\footnotesize
Here are tools to save source codes.
\bit
\item GitHub
\item BitBucket (You can also make private repositories)
\item Gitea\footnote{\texttt{https://gitea.io/en-us/}}
\item GitLab
\bit
\item You can also make private repositories
\item You can import from GitHub, BitBucket, Gitea etc.
\eit
\item Jupyter Notebook (Python, Julia, R are available)
\item Dropbox Paper
\item Slack (Snippets are available, many open source clones are known)
\bit
\item MatterMost
\item Gitter
\item Zulip
\item Rocket.Chat
\eit
\item myCompiler (Compiler on your browser)\footnote{\texttt{https://www.mycompiler.io/}}
\eit
\normalsize
\end{frame}
\section{Tools to Search}
\subsection{Wikipedia Alternatives}
\begin{frame}\ft{\insertsubsection}
\bb{Options}
\bit
\item nLab (about Science)\footnotetext{\texttt{https://ncatlab.org/nlab/show/HomePage}}
\item Scholarpedia (about science)\footnotetext{\texttt{http://www.scholarpedia.org/article/Main\_Page}}
\item Wolfram MathWorld
\item Encyclopedia of Mathematics
\eit
\eb
\end{frame}
\subsection{Q\& A Sites}
\begin{frame}\ft{\insertsubsection}
\footnotesize
\bb{\insertsubsection}
\bit
\item Quora (available in many languages)
\item Stack Exchange (Includes many areas)
\bit
\item Mathematics (for students)/Math Overflow (high level)
\item Mathematica
\item Physics
\item \LaTeX, \TeX
\item Ubuntu
\item Unix \& Linux
\item Computer Science
\item Computational Science
\item Emacs
\item Academia
\item Stack Overflow$\to$Questions about Programming 
\item Cross Validated
\eit
\eit
\eb
\normalsize
\end{frame}
\setbeamerfont{frametitle}{size=\Large}
\begin{frame}\ft{Thank You Very Much}
Your attention and interest is appreciated !!!
\bb{Here's our website}
\footnotesize
\texttt{https://sites.google.com/a/viaprograms.org/via-next-innovators-homepage/}
\normalsize
\eb
\end{frame}
\end{document}
