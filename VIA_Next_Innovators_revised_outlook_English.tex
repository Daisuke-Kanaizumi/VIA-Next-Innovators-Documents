\documentclass[dvipdfmx]{article}
\usepackage[utf8]{inputenc}
\usepackage{amsmath,amsthm,amssymb,amscd,latexsym,tikz}
\usetikzlibrary{shadows}
\usepackage{cases}
\usepackage{ascmac,array}
\usepackage{setspace,subcaption}
\usepackage{fancybox,float}
\usepackage{colortbl,color}
\usepackage{graphicx}
\usepackage{here}
\usepackage{listings}
\usepackage{cite,comment}
\usepackage[top=25truemm,bottom=25truemm,left=30truemm,right=30truemm]{geometry}

\makeatletter
    \renewcommand{\theequation}{%
    \thesection.\arabic{equation}}
    \@addtoreset{equation}{section}
  \makeatother

\title{Revised Outlook\\
VIA Next Innovators}
\author{Made by VIA Next Innovators regular members \& collaborators}
\date{January 2020}
\begin{document}
\maketitle
\newpage
\tableofcontents
\vspace{20truept}
\begin{figure}[h]
\includegraphics{VIANI.jpg}
\end{figure}
\newpage
\section{Backgrounds of Establishment \& Activity Continuation}
Our society is facing numerous difficult social tasks such as religious/ethnic/military confliction, WMD, terrorism, dictatorship, racism, poverty, destruction of nature, natural disasters, lack of effective support for developing countries, human rights protection, etc. Fortunately, some brave activists are already taking actions to solve certain social issues depending on their interest, but the number of change-makers is totally not enough. On the other hand, it is extremely difficult (maybe impossible) for a single person to take action to solve various types of social issues, but that does not mean that he/she can ignore the problems that he/she is not directly related. Therefore, our society is strongly/immediately requiring more change-makers who have passion, vision, knowledge, and compassion to solve social issues.
\par
In July 2015, VIA Next Innovators was founded by Japanese members who completed ESI or DSI (programs focused on social innovation\cite{open,new}, entrepreneurship\cite{social,text,ie,dm}, leadership\cite{lesson,syn}, design-thinking\cite{how,why,research}, etc.) offered by VIA (Volunteers in Asia, or VIA Programs), a non-profit 501(c) organization based in San Francisco, United States. In partnership/cooperation/discussion with VIA, VIA Next Innovators aims at developing/maintaining/expanding a strong professional network of social innovators that will help our society and allow people to reach their potentials. By taking such actions, we will contribute in order to lead the global society in a good way.
\par
As an organization with the \texttt{viaprograms.org} domain, VIA Next Innovators will support and assist VIA.
\section{Vision \& Mission}
\subsection{Vision}
\textcolor{blue}{Our vision is to contribute to our global society by increasing and connecting social innovators in Japan.}\\
(Social Innovators: People who help our society and citizens reach their potential through creative and empathetic action.)
\subsection{Mission}
\begin{itemize}
\item Mission 1: ESI/DSI Participant Recruitment Assistance
\item Mission 2: Alumni Engagement Assistance
\item Mission 3: Take Action
\end{itemize}
\section{Ground Rules}
\subsection{Communication}
\begin{itemize}
\item NO outdebate YES creativity
\item Empathy Listening\cite{power}
\item Responsibility to words
\item Respect of personality
\item Understand the speaker's intent
\end{itemize}
\subsection{Manners}
NEVER forget greetings and punctuality.
\subsection{Information}
\textcolor{blue}{Reinforce information disclosure, distribution \& management.}
\subsection{Mindset}
\begin{itemize}
\item Keep your goals in view
\item Take initiative
\item NEVER forget excitement
\item NO fear to failures
\item Active moves
\end{itemize}

\subsection{Roles}
\begin{itemize}
\item Clarify what to do
\item Maintain equal relations
\item Actively support others\footnote{We will support VIA Programs, former \& future ESI/DSI participants.}
\end{itemize}
\section{Organizational Structure \& Activity Areas}
VIA Next Innovators focuses on the following activity areas:
\begin{itemize}
\item Internet (JPN \& English)
\item Cities in Japan (Headquarters of this activity area is the metropolitan area including Tokyo, and it is known as VIA Next Innovators Tokyo (VIA NIT))
\end{itemize}
VIA Next Innovators is one team including the leader and vice-leader (limited to regular members), and division officers (Public Relations\footnote{Public Relations will control the Main Website, Social Media (Facebook, Instagram, etc.), and knowledge sharing platforms (such as SlideShare, SpeakerDeck, \texttt{slides.com}, GitHub, GitLab, BitBucket, etc.)}, General Affairs, Resource Control\footnote{This division will control budgets and all cybernetic resources.}, Event Planning, Liasion). Local divisions (in any area except Tokyo) may be positioned in case of member enlargements.
\section{Articles of Incorporation}
\subsection{Notes about the Articles of Incorporation}
The Articles of Incorporation has been valid since our establishment. But since 5 years have passed since the kickoff, some parts of the rules do NOT suit to social trends and the current status of VIA Next Innovators. For example, the 1st version was demanding attendance to regular meetings at meeting rooms. But currently there any many ways to hold online meetings (Skype, Slack, Zoom, etc.). We will only show the rules that still suits to our team. We have also fixed English mistakes within the help of Grammarly.
\subsection{Main Text}
\begin{center}
Chapter 1: Name
\end{center}
Article 1: The name of this organization is “VIA Next Innovators.”
\par
\begin{center}
Chapter 2: Purpose and Actions
\end{center}

Article 2: The purpose of VIA Next Innovators is to contribute to our global society by increasing and connecting social innovators in Japan.\\
Article 3: A social innovator is understood as a person who helps our society and supports citizens reach their potential by taking creative and meaningful actions.\\
Article 4: VIA Next Innovators will take the following actions to achieve the purpose listed above\footnote{Order of precedence depends on the decision of active regular members.}.
\begin{enumerate}
\item The organization helps recruit new participants to join VIA’s social innovation programs\footnote{For example, there are ESI/DSI.}.
\item The organization connects past participants of VIA’s social innovation programs and develops new connections among them\footnote{Similar activities are offered by VIA Programs, so we will also assist their events.}.
\item The organization provides opportunities for its members to practice skills\footnote{We are also interested in the following skills that may be related with social innovation:
\begin{itemize}
\item Brainstorming\cite{ctb}
\item Community Organizing\cite{dev,build}
\item Meditation
\item Mindfulness\cite{art,psy}
\end{itemize}
} (e.g., English, design-thinking\cite{how,why,research}) for social innovation.
\item Members of this organization strive to be creative and enjoy the process of maximizing learning.
\end{enumerate}
\begin{center}
Chapter 3: Membership
\end{center}
Article 5: To become a member of VIA Next Innovators, one of the following conditions must be satisfied\footnote{The following conditions are applied to regular members (candidates of the leader \& vice-leader). Guest members, visitors, supporters, cooperators will NOT be restricted by these rules.}:
\begin{enumerate}
\item A person who has completed a social innovation program offered by VIA.
\item A person who has experience managing a social innovation program offered by VIA as a coordinator, fellow, etc\footnote{Program Coordinators, Education Fellows, Alumni Event Managers, Alumni Ambassadors are included.}.
\item A person who has completed a VIA program other than the ones focusing on social innovation\footnote{For example, there are ALC (American Language \& Culture, medical programs, Asian programs, etc.)} can become a member of VIA Next Innovators but he/she has to go through an interview process with the leader of VIA Next Innovators or vice-leader\footnote{If the leaders are absent, then the decision depends on the remaining active regular members.}.
\end{enumerate}
Article 6: Any withdrawal from VIA Next Innovators must be reported.\\
Article 7: If a member meets one of the following conditions, the leader or vice-leader can disaffiliate him/her from the organization:
\begin{enumerate}
\item In case a member violates the law in the Republic of Japan\footnote{We also consider US Federal Laws, Californian Laws (the laws where VIA Programs are located), and VIA's philosophy must NOT be violated.}
\item In case a member violates the articles of incorporation of VIA Next Innovators
\end{enumerate}
\begin{thebibliography}{99}
\bibitem{open}
Murray, R., Caulier-Grice, J., \& Mulgan, G. (2010). The open book of social innovation. London: National endowment for science, technology and the art.
\bibitem{new}
Nicholls, A., Simon, J., Gabriel, M., \& Whelan, C. (Eds.). (2015). New frontiers in social innovation research. Springer.
\bibitem{social}
Mair, J., Robinson, J., \& Hockerts, K. (Eds.). (2006). Social entrepreneurship. New York: Palgrave Macmillan.
\bibitem{text}
Ronstadt, R., \& Robert, R. (1984). Entrepreneurship: Text, cases and notes. Lord Pub..
\bibitem{ie}
Drucker, P. (2014). Innovation and entrepreneurship. Routledge.
\bibitem{md}
Dollinger, M. (2008). Entrepreneurship. Marsh Publications.
\bibitem{lesson}
Hughes, R. L. (1993). Leadership: Enhancing the lessons of experience. Richard D. Irwin, Inc., 1333 Burridge Parkway, Burridge, IL 60521.
\bibitem{syn}
Hunt, J. G. (1991). Leadership: A new synthesis. Sage Publications, Inc.
\bibitem{how}
Cross, N. (2011). Design thinking: Understanding how designers think and work. Berg.
\bibitem{why}
Martin, R., \& Martin, R. L. (2009). The design of business: Why design thinking is the next competitive advantage. Harvard Business Press.
\bibitem{research}
Plattner, H., Meinel, C., \& Leifer, L. (Eds.). (2012). Design thinking research. Springer.
\bibitem{ctb}
Rawlinson, J. G. (2017). Creative thinking and brainstorming. Routledge.
\bibitem{dev}
Rubin, H. J., Rubin, I., \& Doig, R. (1992). Community organizing and development. New York: Macmillan.
\bibitem{build}
Gittell, R., \& Vidal, A. (1998). Community organizing: Building social capital as a development strategy. Sage publications.
\bibitem{art}
Shapiro, S. L., \& Carlson, L. E. (2009). The art and science of mindfulness: Integrating mindfulness into psychology and the helping professions. American Psychological Association.
\bibitem{psy}
Germer, C., Siegel, R. D., \& Fulton, P. R. (Eds.). (2016). Mindfulness and psychotherapy. Guilford Publications.
\bibitem{power}
Rakel, D. (2018). The Compassionate Connection: The Healing Power of Empathy and Mindful Listening. WW Norton \& Company.

\end{thebibliography}
\end{document}