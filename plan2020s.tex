% !TEX TS-program = platex
%!TEX encoding = Shift-JIS
% reference
% https://github.com/munepi/texworks-templates-japanese
\documentclass[dvipdfmx,10pt]{beamer}


\ifnum 42146=\euc"A4A2 %"
\AtBeginShipoutFirst{\special{pdf:tounicode EUC-UCS2}}%
\else
\AtBeginShipoutFirst{\special{pdf:tounicode 90ms-RKSJ-UCS2}}%
\fi

\usefonttheme{professionalfonts}
\setbeamertemplate{navigation symbols}{}
\setbeamertemplate{footline}[frame number]
\setbeamertemplate{bibliography item}[text]
\usepackage{listings}
\usepackage{amsmath,amsfonts,amsthm,latexsym}
\usepackage{lmodern}
\usepackage{pxjahyper}
\usepackage{cases}
\usepackage{comment}
\usepackage{color,colortbl}
\newtheorem{thm}{Theorem}[section]
\newtheorem{df}[thm]{Definition}
\newtheorem{cor}[thm]{Corollary}
\newcommand{\bb}{\begin{block}}
\newcommand{\eb}{\end{block}}
\newcommand{\ft}{\frametitle}
\newcommand{\ssec}{\subsection}
\newcommand{\sss}{\subsubsection}
\newcommand{\insec}{\insertsection}
\newcommand{\iss}{\insertsubsection}
\newcommand{\isss}{\insertsubsubsection}
\newcommand{\bit}{\begin{itemize}}
\newcommand{\eit}{\end{itemize}}

\renewcommand{\familydefault}{\sfdefault}
\renewcommand{\kanjifamilydefault}{\gtdefault}
\usetheme{Berkeley}
\usecolortheme{fly}
\mathversion{bold}
\title{2020年代事業計画 (暫定版)}
\subtitle{Action plan for the 2020s Version 1}
\author[VIA Next Innovators Tokyo]{VIA Next Innovators Tokyo\footnote{\texttt{next\_innovators\_tokyo@viaprograms.org}}}
\date[2020年1月]{2020年1月}
\begin{document}
\begin{frame}
\titlepage
\end{frame}
\begin{frame}{Outline}
%https://tex.stackexchange.com/questions/231128/beamer-highlighting-subsubsections-in-toc
  \setcounter{tocdepth}{3}  
  \tableofcontents[
    sectionstyle=show,
    subsectionstyle=show/show,
    subsubsectionstyle=show/show/show
    ]
\end{frame}
\section{計画概要}
\begin{frame}\ft{\insertsection}
\begin{itemize}
\item 計画は数年後の姿とVision/Missionとの連動を前提に構築する.
\item Vision/Mission は概要書に詳述されている.
\item 数年後の姿は Vision/Mission に準じて定める.
\item 数年後の姿を目指すための戦略も定める.
\item (可能な限り) 各々の戦略を展開する上での具体的目標も作る.
\item 行動指針 (構成員としての在り方) も決める (前回のものを継承)
\item \textcolor{blue}{これらは前回と同様である}
\end{itemize}

\end{frame}
\section{骨子}
\begin{frame}{事業計画骨子 (2020年代)}
\bb{数年後の姿}
\begin{itemize}
\item \textcolor{blue}{(日米の)若者たちの間で Social Innovation $=$ VIA Programs \& VIA Next Innovators という認識が広まっている.}
\item ここでの「日本」は都市部に限定する (離島などは含めない)
\item ここでの「米国」は米国本土の都市部に限定する (Alaska, Hawaii, 及びその他の自治領や海外領土は含めない)
\item 「若者」は16-29歳を想定している.
\item Network の起点となることも目指す.
\end{itemize}
\eb
\end{frame}
\begin{frame}{事業計画骨子 (2020年代)}
\bb{戦略}
\begin{itemize}
\item 日米における VIA Programs \& VIA Next Innovators の認知度/知名度を向上する.
\item 諸活動の活性化を図る.
\item ESI/DSI 参加者水準の維持
\end{itemize}
\eb
\begin{alertblock}{前回との変更点}
増加$\to$維持
\end{alertblock}
\bb{変更理由}
\begin{itemize}
\item \textcolor{blue}{ESI/DSI は量ではなく質を重視している (だから選考がある)}
\item 参加者数をやみくもに増やすことは無益と判断した.
\item そもそも日本と周辺国では少子化が進んでいる.
\end{itemize}
\eb
\end{frame}
\section{具体的目標}
\begin{frame}\ft{\insertsection}
\bb{認知度向上のための目標}
\begin{itemize}
\item \textcolor{blue}{ウェブサイトのアクセス件数が年1000回以上}
\item カウンターの他にも, Google Analytics, Google Search Console, Bing WebMaster Tools などで解析できる.
\item SEO/SEM を徹底する.
\item \textcolor{blue}{SlideShare 上の資料に対する閲覧数が1つにつき100回以上}
\item \textcolor{blue}{Facebook, Instagram などに積極投稿する}
\end{itemize}
\eb
\end{frame}
\begin{frame}\ft{\insertsection}
\begin{itemize}
\item 参加者水準維持のため, 私たちが提供する各種資料の質を向上させる.
\item 活動活性化のため, \textcolor{blue}{社会的に有意義な活動を日本語圏・英語圏のサイバー空間上で展開する}.
\item さらに我々の思想とモデルを国内外に拡大させる.
\end{itemize}
\end{frame}
\section{行動指針}
\setbeamerfont{frametitle}{size=\small}
\begin{frame}\ft{\insertsection}
\footnotesize
\begin{itemize}
\item VIAの一員として誇りと自信を持つ
\item ネットワークを拡大する
\item VIAを正確に知る
\item 熱意とプロ意識を持つ
\item 向上心をもつ
\item オンとオフを使い分ける
\item 責任感を持つ
\item 活動を楽しむ
\item 情報公開と情報発信を徹底する
\item 説明責任を果たす
\item 透明性を担保する
\item 結果を出す
\item チャンスを逃さない
\item 現状に満足しない
\end{itemize}
\normalsize
\end{frame}
\setbeamerfont{frametitle}{size=\Large}
\begin{frame}\ft{\insertsection}
\begin{itemize}
\item 一昔前まで「報告・連絡・相談」が定番だった.
\item \textcolor{blue}{しかしこれこそ人々を束縛していたとも指摘されている.}
\item この手順に縛られないように, 概要書や行動指針ではあえてこの言葉の使用を避けている.
\end{itemize}
\end{frame}
\setbeamerfont{frametitle}{size=\Large}
\begin{frame}\ft{Thank You Very Much}
Your attention and interest is appreciated !!!
\bb{Here's our website}
\footnotesize
\texttt{https://sites.google.com/a/viaprograms.org/via-next-innovators-homepage/}
\normalsize
\eb
\end{frame}
\end{document}